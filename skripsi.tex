%
% Template Laporan Tugas Akhir Jurusan Informatika Unsyiah 
%
% @author Abdul Hafidh
% @version 1.1
% @since 08.09.2023
%
% Template ini telah disesuaikan dengan aturan penulisan tugas akhir yang terdapat pada dokumen Panduan Tugas Akhir FMIPA Unsyiah tahun 2016.
%


% karena jifhasiltheme.cls ada di folder lib, maka kita harus menambahkan path lib/ ke dalam path pencarian file
\makeatletter
\def\input@path{{lib/}}
\makeatother
% Template pembuatan naskah tugas akhir.
\documentclass[dvipsnames]{jifhasiltheme-final}

\tolerance=1
\emergencystretch=\maxdimen
\hyphenpenalty=10000
\hbadness=10000

%\usepackage[a4paper,left=14cm,right=3cm,top=3cm,bottom=5cm]{geometry}
%karena file hype.indonesia.tex ada di folder language, maka kita harus menambahkan path language/ ke dalam path pencarian file
\makeatletter
\def\input@path{{language/}}
\makeatother
% Daftar pemenggalan suku kata dan istilah dalam LaTeX
\include{hype.indonesia}

% Untuk prefiks pada daftar gambar dan tabel
\usepackage[titles]{tocloft}

\usepackage{etoolbox}% http://ctan.org/pkg/etoolbox
\makeatletter
% \patchcmd{<cmd>}{<search>}{<replace>}{<succes>}{<failure>}
\patchcmd{\@chapter}{\addtocontents{lof}{\protect\addvspace{10\p@}}}{}{}{}% LoF
\patchcmd{\@chapter}{\addtocontents{lot}{\protect\addvspace{10\p@}}}{}{}{}% LoT
\makeatother

\usepackage[justification=centering]{caption}% or e.g. [format=hang]

% Ini tambahan dari budi
\newcommand*{\enableboldchapterintoc}{%
  \addtocontents{toc}{\string\renewcommand{\protect\cftchapfont}{\protect\normalfont\protect\bfseries}}
  \addtocontents{toc}{\string\renewcommand{\protect\cftchappagefont}{\protect\normalfont\protect}}
  \addtocontents{toc}{\protect\setlength{\cftbeforechapskip}{12pt}}
}
\newcommand*{\disableboldchapterintoc}{%
  \addtocontents{toc}{\string\renewcommand{\protect\cftchappagefont}{\protect\normalfont}}
  \addtocontents{toc}{\string\renewcommand{\protect\cftchapfont}{\protect\normalfont}}
  \addtocontents{toc}{\protect\setlength{\cftbeforechapskip}{0pt}}
}
% Ini tambahan dari budi

\renewcommand{\cftdotsep}{0.5}
\renewcommand{\cftchapleader}{\cftdotfill{\cftdotsep}}
%\renewcommand{\cftpartleader}{\cftdotfill{\cftdotsep}}

\renewcommand\cftfigpresnum{Gambar\  }
\renewcommand\cfttabpresnum{Tabel\   }

\newcommand{\listappendicesname}{DAFTAR LAMPIRAN}
\newlistof{appendices}{apc}{\listappendicesname}
\newcommand{\appendices}[1]{\addcontentsline{apc}{appendices}{#1}}
\newcommand{\newappendix}[1]{\section*{#1}\appendices{#1}}

% Untuk hyperlink dan table of content
\usepackage[hidelinks]{hyperref}
\renewcommand\UrlFont{\rmfamily\itshape} %it's me!
\newlength{\mylenf}
\settowidth{\mylenf}{\cftfigpresnum}
\setlength{\cftfignumwidth}{\dimexpr\mylenf+2em}
\setlength{\cfttabnumwidth}{\dimexpr\mylenf+2em}

% Agar ada tulisan BAB pada TOC
\renewcommand\cftchappresnum{BAB } 
  \cftsetindents{chapter}{0em}{4.5em} %indenting bab
  \cftsetindents{section}{4.5em}{2em}
  \cftsetindents{subsection}{6.5em}{3em}
 
% Agar di TOC setiap angka bab/subbab diakhiri titik

\renewcommand{\cftsecaftersnum}{.}
\renewcommand{\cftsubsecaftersnum}{.}

\addtocontents{toc}{~\hfill \textit{Halaman}\par} % Menambahkan kata "Halaman" di sebelah kanan daftar isi

% Membuat juga kata "halaman" di daftar gambar dan daftar tabel
\addtocontents{lof}{~\hfill \textit{Halaman}\par}
\addtocontents{lot}{~\hfill \textit{Halaman}\par}

% Membuat juga kata "halaman" di daftar lampiran
\addtocontents{apc}{~\hfill \textit{Halaman}\par}

% Agar setiap angka bab/subbab diakhiri titik
\usepackage{titlesec}
\titlelabel{\thetitle.\quad}

% Agar disetiap caption table dan gambar diakhiri titik
\usepackage[labelsep=period]{caption}

% Untuk Bold Face pada Keterangan Gambar
\usepackage[labelfont=bf]{caption}

% Untuk caption dan subcaption
\usepackage{caption}
\usepackage{subcaption}


% Agar bisa menggunakan warna LaTeX
\usepackage{color} %it's me!

% Agar table yang panjang bisa cut ke next page    %byRennyAdr
\usepackage{longtable}

% Untuk page landscape        %byRennyAdr
\usepackage{pdflscape}
\usepackage{lscape}

% Agar bisa bikin code snippet
\usepackage{listings, lstautogobble} %it's me!

\usepackage{adjustbox}

% untuk shadow gambar     %tomy
\usepackage{fancybox, graphicx}

% untuk cite url
\usepackage{url}
\usepackage{microtype} % untuk mengatur spasi pada paragraf

\usepackage{siunitx}

\usepackage{xcolor}
\usepackage{multirow}
\usepackage[normalem]{ulem}
\useunder{\uline}{\ul}{}

\usepackage{array}
\newcolumntype{P}[1]{>{\centering\arraybackslash}p{#1}}
\newcolumntype{M}[1]{>{\centering\arraybackslash}m{#1}}


\makeatletter
\def\input@path{{include/}}
\makeatother
% Sampul Depan
%-----------------------------------------------------------------
% Sampul Depan
%-----------------------------------------------------------------
\judulcover{LOREM IPSUM DOLOR SI AMET}

\judul{LOREM IPSUM DOLOR SI AMET}

\judulinggris{LOREM IPSUM DOLOR SI AMET}

% nama lengkap
\fullname{Anton }

% NPM (Nomor Pokok Mahasiswa)
\idnum{2134312432134}

\degree{Sarjana Komputer}

\yearsubmit{Januari, 2024}

\program{Informatika}

\dept{Informatika}

% Pembimbing Pertama
\firstsupervisor{<pembimbing1>}
\firstnip{<nip pembimbing1}

% Pembimbing Kedua
\secondsupervisor{<pembimbing 2>}
\secondnip{<nip pembimbing 2>}

% Ketua Jurusan
\kajur{Viska Mutiawani, B.IT, M.IT.}
\kajurnip{198008312009122003}

% Dekan Fakultas
\dekan{Prof. Dr. Teuku M. Iqbalsyah, S.Si, M.Sc.}
\dekannip{197110101997031003}
% Dekan Fakultas
%\dekan{Dr. Teuku Mohamad Iqbalsyah, S.Si., M.Sc.}
%\dekannip{197110101997031003}

%kaprodi
\kaprodi{Viska Mutiawani, B.IT, M.IT.}
\kaprodinip{198008312009122003}

% tangal lulus proposal, seminar hasil atau sidang
\approvaldate{Kamis, 14 Mei 2024}

%-----------------------------------------------------------------
% End of Sampul Depan
%-----------------------------------------------------------------


% Awal dokumen
\usepackage{fancyhdr}
\usepackage{rotating}
% Untuk prefiks pada Daftar Program   
% byRennyAdr
\makeatletter
\begingroup\let\newcounter\@gobble\let\setcounter\@gobbletwo
\globaldefs\@ne \let\c@loldepth\@ne
\newlistof{listings}{lol}{\lstlistlistingname}
\endgroup
\let\l@lstlisting\l@listings
\AtBeginDocument{\addtocontents{lol}{\protect\addvspace{10\p@}}}
\makeatother
\renewcommand{\lstlistoflistings}{\listoflistings}
\renewcommand\cftlistingspresnum{Program~}
\cftsetindents{listings}{1.5em}{7em}

%tab didaftar pustaka -Indah
\setlength{\bibhang}{30pt}

%split rumus -Indah
\usepackage{amsmath}
\usepackage{pdfpages}

% \usepackage{multirow}
\usepackage[table,xcdraw]{xcolor}
\usepackage{colortbl}
% Beamer presentation requires  instead of \usepackage[table,xcdraw]{xcolor}
% \usepackage{longtable}

\begin{document}
\fancyhf{} 
\fancyfoot[C]{\thepage}



\cover

\approvalpage

\bplagiasi % Note: \preface JANGAN DIHAPUS!

\noindent
Saya yang bertanda tangan di bawah ini,

\vspace{-0.1cm}

\begin{table}[H]
\begin{tabular}{M{0.6cm}ll}
	&Nama lengkap   		&: Abdul Hafidh \\
	&Tempat/tanggal lahir	&: Banda Aceh/ 29 Maret 2002 \\
	&NPM       			&: 2008107010056    \\
	&Program Studi   		&: Informatika \\
	&Fakultas 				&: MIPA \\
	&Judul Tugas Akhir      &: \begin{tabularx}{\linewidth}[t]{@{}X@{}}
		PENERAPAN \textit{VISUAL QUESTION ANSWERING} DALAM \\
		MENAFSIRKAN CITRA MEDIS MENGGUNAKAN \\
		\textit{DEEP LEARNING}
	   \end{tabularx}
	% &Judul Tugas Akhir      &: PENERAPAN \textit{VISUAL QUESTION ANSWERING} DALAM \\
	% &						&   MENAFSIRKAN CITRA MEDIS MENGGUNAKAN \\
	% &						&  \textit{DEEP LEARNING}
\end{tabular}
\end{table}

\vspace{0.2cm}
\noindent
Menyatakan dengan sesungguhnya bahwa Laporan Tugas Akhir saya dengan judul di atas adalah \textbf{hasil karya saya sendiri} bersama dosen pembimbing dan \textbf{bebas plagiasi}.

\vspace{1cm}
\noindent
Jika ternyata di kemudian hari terbukti bahwa Laporan Tugas Akhir merupakan hasil plagiasi, saya bersedia menerima sanksi yang berlaku di Universitas Syiah Kuala.

\vspace{1cm}


\begin{tabular}{p{7.5cm}l}
	&Banda Aceh, 14 Januari 2024\\
	&\\
	&\\
	&\multirow{1.5}{7.5cm}{\underline{Anton}} \\ 
	&NPM. 2342342342342 \\
\end{tabular}
\spernyataan % Note: \preface JANGAN DIHAPUS!

\noindent
Yang bertanda tangan di bawah ini,
\vspace{-0.1cm}
\begin{table}[H]
{\renewcommand{\arraystretch}{0.7}
\begin{tabular}{M{0.6cm}lll}
	&1. 	& Nama   		&: Anton \\
	&	& NPM       			&: 2134312432134   \\
	&	& Jurusan/Prodi   		&: Informatika \\
	&	& Status 				&: Mahasiswa \\  
	&2. 	& Nama  		&: <pembimbing1> \\
	&	& NIP       			&: <nip pembimbing1>   \\
	&	& Jurusan/Prodi   		&: Informatika \\
	&	& Status 				&: Pembimbing I \\  
	&3. 	& Nama  		&: <pembimbing 2> \\
	&	& NIP       			&: <nip pembimbing 2>   \\
	&	& Jurusan/Prodi   		&: Informatika \\
	&	& Status 				&: Pembimbing II   
\end{tabular}
}
\end{table}
\vspace{-0.4cm}
\noindent
Dengan ini menyatakan hasil penelitian Tugas Akhir yang berjudul \textbf{“PENERAPAN \textit{VISUAL QUESTION ANSWERING} DALAM MENAFSIRKAN CITRA MEDIS MENGGUNAKAN \textit{DEEP LEARNING}”} tidak dipublikasikan secara \textit{full-text} di sistem ETD (\textit{Electronic Theses and Dissertations}) Universitas Syiah Kuala hingga batas waktu 5 tahun dari tanggal kelulusan.

\vspace{0.4cm}
\noindent
Demikian surat pernyataan ini dibuat dengan sebenarnya untuk dapat dipergunakan seperlunya.

\vspace{0.4cm}
{\renewcommand{\arraystretch}{0.8}
\centering
\begin{tabular}{lll}
	&Darussalam, 14 Mei 2022		& \\
	&Yang membuat pernyataan,			& \\
	&&\\
	Pembimbing I,							&Pembimbing II,							&Mahasiswa,\\
	&&\\
	&&\\
	&&\\
	\underline{Alim Misbullah, S.Si., M.S.}	&\underline{<pembimbing 2} &\underline{Abdul Hafidh}\\
	NIP. 198806032019031011				&NIP. <nip pembimbing 2>				&NPM. 2008107010056\\
	&&\\
	&Mengetahui:\\			&
\end{tabular}
}
{\renewcommand{\arraystretch}{0.8}
\begin{tabular}{lll}
	Koordinator Program Studi Informatika	&\qquad\qquad  &Koordinator TA,\\
	Universitas Syiah Kuala,&\quad\quad  &\\
	&&\\
	&&\\
	&&\\
	\underline{Viska Mutiawani, B.IT, M.IT.}	&\quad\quad  &\underline{Alim Misbullah, S.Si., M.S.}\\
	NIP. 198008312009122003						&\quad\quad  &NIP. 198806032019031011				
\end{tabular}
}

\include{abstrak-indonesia} %berikan comment jika proposal

\begin{abstracteng}
\textit{lorem ipsum dolot sit amet, consectetur adipiscing elit. Sed euismod, nisl quis lacinia ultricies, nunc nisl ultricies diam, quis aliquam nisl nisl quis nisl.}

\bigskip
\noindent
\textbf{Kata kunci :} lorem, ipsum, dolor, sit, amet
\end{abstracteng} %berikan comment jika proposal
%-----------------------------------------------------------------
% Disini kata pengantar
%-----------------------------------------------------------------
\preface % Note: \preface JANGAN DIHAPUS!


Segala puji dan syukur kehadiran Allah SWT yang telah melimpahkan rahmat dan hidayah-Nya kepada kita semua, sehingga penulis dapat menyelesaikan penulisan Tugas Akhir yang berjudul \textbf{“PENERAPAN \textit{VISUAL QUESTION ANSWERING} DALAM MENAFSIRKAN CITRA MEDIS MENGGUNAKAN \textit{DEEP LEARNING}”} yang telah dapat diselesaikan sesuai rencana. Penulis banyak mendapatkan berbagai pengarahan, bimbingan, dan bantuan dari berbagai pihak. Oleh karena itu, melalui tulisan ini penulis mengucapkan rasa terima kasih kepada:

\begin{enumerate}
	\item{Ayah dan Ibu sebagai kedua orang tua penulis yang senantiasa selalu mendukung aktivitas dan kegiatan yang penulis lakukan baik secara moral maupun material serta menjadi motivasi terbesar bagi penulis untuk menyelesaikan Tugas Akhir ini.}
	\item{Bapak Alim Misbullah, S.Si., M.S. selaku Dosen Pembimbing I yang telah banyak memberikan bimbingan dan arahan kepada penulis, sehingga penulis dapat menyelesaikan Tugas Akhir ini.}
	\item{Bapak/Ibu <pembimbing 2> selaku Dosen Pembimbing II yang telah banyak memberikan bimbingan dan arahan kepada penulis, sehingga penulis dapat menyelesaikan Tugas Akhir ini.}
	\item{Ibu Dalila Husna Yunardi, B.Sc., M.Sc., selaku Dosen Wali.}
	\item {Ibu Viska Mutiawani, B.IT, M.IT., selaku Ketua Program Studi Informatika dan Dosen Penguji Tugas Akhir I.}
	\item{Seluruh Dosen di Jurusan Informatika Fakultas MIPA atas ilmu dan didikannya selama perkuliahan.}
	\item{Sahabat dan teman-teman seperjuangan Jurusan Informatika USK 2020 lainnya.}
\end{enumerate}

%\vspace{1.5cm}

Penulis juga menyadari segala ketidaksempurnaan yang terdapat didalamnya baik dari segi materi, cara, ataupun bahasa yang disajikan. Seiring dengan ini penulis mengharapkan kritik dan saran dari pembaca yang sifatnya dapat berguna untuk kesempurnaan Tugas Akhir ini. Harapan penulis semoga tulisan ini dapat bermanfaat bagi banyak pihak dan untuk perkembangan ilmu pengetahuan.

\vspace{1cm}


\begin{tabular}{p{7.5cm}l}
	&Banda Aceh, 14 Januari 2024\\
	&\\
	&\\
	&\multirow{1.5}{7.5cm}{\underline{Anton}} \\ 
	&NPM. 2342343242 \\
\end{tabular}

\includepdf[pages=-]{ttd-seminar-hasil_signed.pdf}

\titleformat{\section}{\normalfont\bfseries\uppercase}{\thesection}{1.7em}{} % Untuk membuat section menjadi kapital tapi daftar isi tetap non kapital

% Untuk subsection set saja jaraknya 1.5em (ga perlu buat kapital seperti section)
\titleformat{\subsection}{\normalfont\bfseries}{\thesubsection}{0.9em}{}




%-----------------------------------------------------------------
% TOC menggunakan single space
%-----------------------------------------------------------------

\addcontentsline{toc}{chapter}{Daftar Isi}
\begin{singlespace}
	\tableofcontents
\end{singlespace}
\listoftables
\addcontentsline{toc}{chapter}{Daftar Tabel}
\listoffigures
\addcontentsline{toc}{chapter}{Daftar Gambar}

% \renewcommand{\lstlistlistingname}{DAFTAR PROGRAM}
% \lstlistoflistings
% \addcontentsline{toc}{chapter}{Daftar Program}

\listofappendices
\addcontentsline{toc}{chapter}{Daftar Lampiran}

\enableboldchapterintoc
%-----------------------------------------------------------------
% Daftar Singkatan 
%-----------------------------------------------------------------
\include{daftar-singkatan}

% Caption untuk code snippet. it's me!
\renewcommand{\thelstlisting}{\arabic{chapter}.\arabic{lstlisting}}
\renewcommand*\lstlistingname{Program}

%-----------------------------------------------------------------
% Disini awal masukan untuk Bab
%-----------------------------------------------------------------
\begin{onehalfspace}

\fancyhf{} 
\fancyfoot[C]{\thepage}
\pagenumbering{arabic}

\captionsetup[figure]{labelfont={normalfont}, textfont={normalfont}} % Unbold caption gambar
\captionsetup[table]{labelfont={normalfont}, textfont={normalfont}} % Unbold caption tabel

\fancyhf{} 
\fancyfoot[R]{\thepage}

% \fancyhf{} 
% \fancyfoot[R]{\thepage}

\chapter{PENDAHULUAN}
%\thispagestyle{plain} % Halaman pertama bab menggunakan gaya plain

\section{Latar Belakang}
% Menambahkan lorem ipsum
Lorem ipsum dolor sit amet, consectetur adipiscing elit. Sed euismod, nisl quis lacinia ultricies, nunc nisl ultricies diam, quis aliqua \citep{WHO2020}


\section{Rumusan Masalah}
Berdasarkan latar belakang di atas, permasalahan dalam penelitian ini dapat dirumuskan sebagai berikut:
\begin{enumerate}
	\item lorem ipsum dolor sit amet, consectetur adipiscing elit. Sed euismod, nisl quis lacinia ultricies,
	\item lorem ipsum dolor sit amet, consectetur adipiscing elit. Sed euismod, nisl quis lacinia ultricies,
	\item lorem ipsum dolor sit amet, consectetur adipiscing elit. Sed euismod, nisl quis lacinia ultricies,
	\item lorem ipsum dolor sit amet, consectetur adipiscing elit. Sed euismod, nisl quis lacinia ultricies,
\end{enumerate}

\section{Tujuan Penelitian}
Berdasarkan rumusan masalah yang telah disebutkan sebelumnya, maka dapat dipaparkan tujuan dari penelitian ini adalah sebagai berikut:
\begin{enumerate}
	\item lorem ipsum dolor sit amet, consectetur adipiscing elit. Sed euismod, nisl quis lacinia ultricies,
	\item lorem ipsum dolor sit amet, consectetur adipiscing elit. Sed euismod, nisl quis lacinia ultricies,
	\item lorem ipsum dolor sit amet, consectetur adipiscing elit. Sed euismod, nisl quis lacinia ultricies,
	\item lorem ipsum dolor sit amet, consectetur adipiscing elit. Sed euismod, nisl quis lacinia ultricies,
\end{enumerate}


\section{Manfaat Penelitian}
Adapun manfaat dari penelitian ini adalah sebagai berikut:
\begin{enumerate}
	\item lorem ipsum dolor sit amet, consectetur adipiscing elit. Sed euismod, nisl quis lacinia ultricies,
	\item lorem ipsum dolor sit amet, consectetur adipiscing elit. Sed euismod, nisl quis lacinia ultricies,
	\item lorem ipsum dolor sit amet, consectetur adipiscing elit. Sed euismod, nisl quis lacinia ultricies,
	\item lorem ipsum dolor sit amet, consectetur adipiscing elit. Sed euismod, nisl quis lacinia ultricies,

\end{enumerate}



% Baris ini digunakan untuk membantu dalam melakukan sitasi
% Karena diapit dengan comment, maka baris ini akan diabaikan
% oleh compiler LaTeX.
\begin{comment}
\bibliography{daftar-pustaka}
\end{comment}


\fancyhf{} 
\fancyfoot[R]{\thepage}

%-------------------------------------------------------------------------------
%                            BAB II
%               TINJAUAN PUSTAKA DAN DASAR TEORI
%-------------------------------------------------------------------------------
\fancyhf{} 
\fancyfoot[R]{\thepage}
\chapter{TINJAUAN PUSTAKA}

\par Untuk mendukung penelitian ini, maka dalam bab ini akan dikemukakan beberapa rumusan teori pendukung yang dikutip dari berbagai referensi baik dalam bentuk buku, artikel, maupun tulisan karya ilmiah lainnya termasuk hasil penelitian sebelumnya yang ada kaitannya dengan penelitian yang dilakukan.
\section{Landasan Teori}
%\subsection{System Usability Scale (SUS)}	
%\par \textit{System Usability Scale} (SUS) dikembangkan oleh \citep{brooke1996} sebagai sebuah pengukuran usability yang \textit{quick and dirty}. SUS menggunakan survei yang terdiri dari 10 pertanyaan, masing-masing memiliki 5 poin Likert sebagai tanggapan, output dari SUS berupa skor yang tampak mudah dipahami, dengan \textit{range} dari 0 hingga 100. Semakin besar skor berarti semakin baik \textit{usability}-nya. 

% \par Pengujian ini memerlukan kuesioner sebagai teknik pengumpulan informasi pengujian yang didapat dari responden mengenai aplikasi. Kuesioner terdiri dari 10 pertanyaan yang terdiri dari pertanyaan positif dan negatif. Pertanyaan positif terdapat pada nomor ganjil (1, 3, 5, 7, 9) dan pertanyaan negatif terdapat pada nomor genap (2, 4, 6, 8, 10). Setiap pertanyaan diberi bobot antara 0 - 4. Pertanyaan ganjil skor dihitung dengan cara bobot tiap pertanyaan (x\textsubscript{i}) dikurangi 1 (ditulis dengan x\textsubscript{i} - 1). Sedangkan pertanyaan genap skor dihitung dengan cara 5 dikurangi bobot tiap pertanyaan (x\textsubscript{i}) (ditulis dengan 5 - x\textsubscript{i}) \citep{ardiansyah2016}.

%-----------------------------------------------------------------------------%

% Baris ini digunakan untuk membantu dalam melakukan sitasi
% Karena diapit dengan comment, maka baris ini akan diabaikan
% oleh compiler LaTeX.

\section{Arsitektur FaceNet}
\par \textit{FaceNet} merupakan \textit{Convolutional Neural Network} (CNN) yang digunakan untuk pengenalan wajah, dikembangkan oleh peneliti dari Google dan dikenalkan pada tahun 2015 \citep{jose2019}. \textit{FaceNet} digunakan untuk mengekstraksi fitur dari gambar wajah seseorang. \textit{FaceNet} mengekstrak wajah menjadi vektor menggunakan \textit{deep} CNN. Vektor nilai atau \textit{vector embedding} yang dihasilkan dapat memetakan kemiripan wajah yang memiliki kedekatan posisi pada \textit{embedding space} \citep{rajagede2021}.

% Menambahkan gambar 
\begin{figure}[H]
\centering
\frame{\includegraphics [width = 14cm, height= 5cm]{image/diagram_facenet}}
\caption{\textit{Desain} Arsitektur \textit{FaceNet}}.
\label{dig_facenet}
\end{figure}

\begin{figure}[H]
\centering
\frame{\includegraphics [width = 14cm, height= 5cm]{image/diagram_facenet}}
\caption{\textit{Desain} Arsitektur \textit{FaceNet}}.
\label{dig_facenet}
\end{figure}

\fancyhf{} 
\fancyfoot[R]{\thepage}

\begin{comment}
\bibliography{daftar-pustaka}
\end{comment}


\fancyhf{} 
\fancyfoot[R]{\thepage}

%-------------------------------------------------------------------------------
%                            BAB III
%               		METODOLOGI PENELITIAN
%-------------------------------------------------------------------------------
\fancyhf{} 
\fancyfoot[R]{\thepage}
\chapter{METODE PENENILITIAN}
\section{Waktu dan Lokasi Penelitian}
Penelitian ini akan bertempat pada beberapa ruangan yang digunakan oleh mahasiswa Jurusan Informatika USK yang umumnya terletak pada lantai 3 blok A dan blok E Gedung Fakultas MIPA USK. Waktu yang dibutuhkan agar penelitian ini dapat di implementasikan adalah 4 bulan terhitung dari bulan Januari 2024 hingga Mei 2024.

\section{Alat dan Bahan}
Alat dan Bahan yang akan digunakan pada penelitian ini terdiri dari beberapa perangkat keras (\textit{hardware}) dan perangkat lunak (\textit{software}) yang dijabarkan sebagai berikut:

\begin{enumerate}
\item Perangkat Keras
	\begin{itemize}
	\item Laptop Lenovo Yoga C740 dengan RAM 16GB DDR4, Intel® Core™ i7-10710U. 1.10 - 4.70 GHz,\textit{Solid State State Drive} (SSD) 1TB.
	\end{itemize}

\item Perangkat Lunak
	\begin{itemize}
	\item Linux Debian Ubuntu 22.04 LTS
	\item Visual Studio Code 1.71.0
	\item Python 3.8.17
	\end{itemize}
\end{enumerate}
\newpage
\section{Metode Penelitian}
Metode penelitian yang dilakukan akan terdiri dari beberapa tahapan penelitian. Skema dari alur tahapan tersebut dapat dilihat pada Gambar \ref{alur_penelitian}.

%-----------------------------------------------------------------------------%

% Baris ini digunakan untuk membantu dalam melakukan sitasi
% Karena diapit dengan comment, maka baris ini akan diabaikan
% oleh compiler LaTeX.
\begin{comment}
\bibliography{daftar-pustaka}
\end{comment}

\fancyhf{} 
\fancyfoot[R]{\thepage}

%-------------------------------------------------------------------------------
%                            BAB IV
%               		HASIL DAN PEMBAHASAN
%-------------------------------------------------------------------------------
\fancyhf{} 
\fancyfoot[R]{\thepage}
\chapter{HASIL DAN PEMBAHASAN}



% Baris ini digunakan untuk membantu dalam melakukan sitasi
% Karena diapit dengan comment, maka baris ini akan diabaikan
% oleh compiler LaTeX.
\begin{comment}
\bibliography{daftar-pustaka}
\end{comment}

\fancyhf{} 
\fancyfoot[R]{\thepage}

%-------------------------------------------------------------------------------
%                            BAB V
%               		KESIMPULAN DAN SARAN
%-------------------------------------------------------------------------------
% \fancyhf{} 
% \fancyfoot[R]{\thepage}
\chapter{KESIMPULAN DAN SARAN}
%\thispagestyle{plain} % Halaman pertama bab menggunakan gaya plain

\section{Kesimpulan}

\section{Saran}


%-----------------------------------------------------------------------------%

% Baris ini digunakan untuk membantu dalam melakukan sitasi
% Karena diapit dengan comment, maka baris ini akan diabaikan
% oleh compiler LaTeX.
\begin{comment}
\bibliography{daftar-pustaka}
\end{comment}


\fancypagestyle{daftarpustaka}{
    \fancyhf{} % Hapus semua header dan footer yang sudah ada
    \fancyfoot[R]{\thepage} % Letakkan nomor halaman di tengah kepala (center)
    \renewcommand{\headrulewidth}{0pt} % Hapus garis pemisah kepala dengan konten
    \renewcommand{\footrulewidth}{0pt} % Hapus garis pemisah footer dengan konten
}


% Halaman Daftar Pustaka dengan penomoran halaman berada di tengah dan nomor halaman selanjutnya di kanan
\addcontentsline{toc}{chapter}{DAFTAR PUSTAKA}
\begin{onehalfspace}
\begin{spacing}{1}
\pagestyle{daftarpustaka}
\bibliography{daftar-pustaka}
\end{spacing}
%-----------------------------------------------------------------
% Disini akhir masukan Daftar Pustaka
%-----------------------------------------------------------------

%%
% @author Kurnia Saputra
% @version 1.0
% 
% Hanya sebuah pembatas bertuliskan LAMPIRAN ditengah halaman. 
% 

\begin{titlepage}
	\centering 
	\vspace*{6cm}
	\noindent \Huge{LAMPIRAN}
	%\addChapter{LAMPIRAN}
	\addcontentsline{toc}{chapter}{LAMPIRAN}
\end{titlepage}
\addcontentsline{toc}{chapter}{LAMPIRAN}
\chapter*{LAMPIRAN}

\addcontentsline{toc}{chapter}{LAMPIRAN} %daftar lampiran

\end{onehalfspace}

\end{document}