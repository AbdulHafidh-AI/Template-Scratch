\fancyhf{} 
\fancyfoot[R]{\thepage}

\chapter{PENDAHULUAN}

\section{Latar Belakang}
% Menambahkan lorem ipsum
Lorem ipsum dolor sit amet, consectetur adipiscing elit. Sed euismod, nisl quis lacinia ultricies, nunc nisl ultricies diam, quis aliqua

\subsection{Yanti}
bla bla bla bla bla


\section{Rumusan Masalah}
Berdasarkan latar belakang di atas, permasalahan dalam penelitian ini dapat dirumuskan sebagai berikut:
\begin{enumerate}
	\item lorem ipsum dolor sit amet, consectetur adipiscing elit. Sed euismod, nisl quis lacinia ultricies,
	\item lorem ipsum dolor sit amet, consectetur adipiscing elit. Sed euismod, nisl quis lacinia ultricies,
	\item lorem ipsum dolor sit amet, consectetur adipiscing elit. Sed euismod, nisl quis lacinia ultricies,
	\item lorem ipsum dolor sit amet, consectetur adipiscing elit. Sed euismod, nisl quis lacinia ultricies,
\end{enumerate}

\section{Tujuan Penelitian}
Berdasarkan rumusan masalah yang telah disebutkan sebelumnya, maka dapat dipaparkan tujuan dari penelitian ini adalah sebagai berikut:
\begin{enumerate}
	\item lorem ipsum dolor sit amet, consectetur adipiscing elit. Sed euismod, nisl quis lacinia ultricies,
	\item lorem ipsum dolor sit amet, consectetur adipiscing elit. Sed euismod, nisl quis lacinia ultricies,
	\item lorem ipsum dolor sit amet, consectetur adipiscing elit. Sed euismod, nisl quis lacinia ultricies,
	\item lorem ipsum dolor sit amet, consectetur adipiscing elit. Sed euismod, nisl quis lacinia ultricies,
\end{enumerate}


\section{Manfaat Penelitian}
Adapun manfaat dari penelitian ini adalah sebagai berikut:
\begin{enumerate}
	\item lorem ipsum dolor sit amet, consectetur adipiscing elit. Sed euismod, nisl quis lacinia ultricies,
	\item lorem ipsum dolor sit amet, consectetur adipiscing elit. Sed euismod, nisl quis lacinia ultricies,
	\item lorem ipsum dolor sit amet, consectetur adipiscing elit. Sed euismod, nisl quis lacinia ultricies,
	\item lorem ipsum dolor sit amet, consectetur adipiscing elit. Sed euismod, nisl quis lacinia ultricies,

\end{enumerate}



% Baris ini digunakan untuk membantu dalam melakukan sitasi
% Karena diapit dengan comment, maka baris ini akan diabaikan
% oleh compiler LaTeX.
\begin{comment}
\bibliography{daftar-pustaka}
\end{comment}